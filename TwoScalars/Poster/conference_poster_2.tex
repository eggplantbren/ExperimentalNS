%%%%%%%%%%%%%%%%%%%%%%%%%%%%%%%%%%%%%%%%%
% Dreuw & Deselaer's Poster
% LaTeX Template
% Version 1.0 (11/04/13)
%
% Created by:
% Philippe Dreuw and Thomas Deselaers
% http://www-i6.informatik.rwth-aachen.de/~dreuw/latexbeamerposter.php
%
% This template has been downloaded from:
% http://www.LaTeXTemplates.com
%
% License:
% CC BY-NC-SA 3.0 (http://creativecommons.org/licenses/by-nc-sa/3.0/)
%
%%%%%%%%%%%%%%%%%%%%%%%%%%%%%%%%%%%%%%%%%

%----------------------------------------------------------------------------------------
%	PACKAGES AND OTHER DOCUMENT CONFIGURATIONS
%----------------------------------------------------------------------------------------

\documentclass[final,hyperref={pdfpagelabels=false}]{beamer}

\usepackage[orientation=portrait,size=a0,scale=1.4]{beamerposter} % Use the beamerposter package for laying out the poster with a portrait orientation and an a0 paper size

\usetheme{I6pd2} % Use the I6pd2 theme supplied with this template

\usepackage[english]{babel} % English language/hyphenation

\usepackage{amsmath,amsthm,amssymb,latexsym} % For including math equations, theorems, symbols, etc

%\usepackage{times}\usefonttheme{professionalfonts}  % Uncomment to use Times as the main font
%\usefonttheme[onlymath]{serif} % Uncomment to use a Serif font within math environments

\boldmath % Use bold for everything within the math environment

\usepackage{booktabs} % Top and bottom rules for tables

\graphicspath{{figures/}} % Location of the graphics files

\usecaptiontemplate{\small\structure{\insertcaptionname~\insertcaptionnumber: }\insertcaption} % A fix for figure numbering

%----------------------------------------------------------------------------------------
%	TITLE SECTION 
%----------------------------------------------------------------------------------------

\title{\huge Nested sampling with multiple scalars} % Poster title

\author{Brendon J. Brewer} % Author(s)

\institute{Department of Statistics, The University of Auckland} % Institution(s)

%----------------------------------------------------------------------------------------
%	FOOTER TEXT
%----------------------------------------------------------------------------------------

\newcommand{\leftfoot}{http://www.stat.auckland.ac.nz/~brewer/} % Left footer text

\newcommand{\rightfoot}{bj.brewer@auckland.ac.nz} % Right footer text

%----------------------------------------------------------------------------------------

\begin{document}

\addtobeamertemplate{block end}{}{\vspace*{2ex}} % White space under blocks

\begin{frame}[t] % The whole poster is enclosed in one beamer frame

\begin{columns}[t] % The whole poster consists of two major columns, each of which can be subdivided further with another \begin{columns} block - the [t] argument aligns each column's content to the top

\begin{column}{.02\textwidth}\end{column} % Empty spacer column

\begin{column}{.465\textwidth} % The first column

%----------------------------------------------------------------------------------------
%	OBJECTIVES
%----------------------------------------------------------------------------------------

\begin{block}{Abstract}

Nested Sampling is an effective and popular
Monte Carlo algorithm for Bayesian computation. It is based on exploring the
prior $\pi(\mathbf{x})$ and successively imposing constraints on the value of
the likelihood $\mathcal{L}(\mathbf{x})$ that compress the prior mass by a
``known'' factor. This enables the calculation of the marginal likelihood
or evidence
$\mathcal{Z} = \int \pi(\mathbf{x})\mathcal{L}(\mathbf{x}) \, d^N\mathbf{x}$ and
properties of the posterior, or any other distribution that is intermediate
between the prior and the posterior. However, in some inference and
statistical mechanics problems, there are two or more scalar functions of
$\mathbf{x}$ that are relevant. For example, we might be interested in the
properties of
$p(\mathbf{x} | \lambda_1, \lambda_2) \propto \pi(\mathbf{x})\exp\left[-\lambda_1 f_1(\mathbf{x}) - \lambda_2 f_2(\mathbf{x})\right]$
for many different values of $\lambda_1$ and $\lambda_2$, including the
normalisations or ``partition function'' $\mathcal{Z}(\lambda_1, \lambda_2)$.
This work describes
progress towards solving this class of problems while maintaining the benefits
of Nested Sampling, such as the ability to cope with first-order phase
transitions.

\end{block}

%----------------------------------------------------------------------------------------
%	INTRODUCTION
%----------------------------------------------------------------------------------------
            
\begin{block}{Introduction}

In statistical mechanics, a family of ``canonical distributions'' is given by
\begin{eqnarray}
p(\mathbf{x} | \lambda_1, \lambda_2) \propto
\pi(\mathbf{x})
\exp\left[-\lambda_1 f_1(\mathbf{x}) - \lambda_2 f_2(\mathbf{x})\right]
\end{eqnarray}
The normalisation, or ``partition function'', depends on the $\lambda$
parameters:

\begin{eqnarray}
\mathcal{Z} = \int \pi(\mathbf{x})
\exp\left[-\lambda_1 f_1(\mathbf{x}) - \lambda_2 f_2(\mathbf{x})\right]
\, d^N \mathbf{x}
\end{eqnarray}

\end{block}

%----------------------------------------------------------------------------------------
%	MATERIALS
%----------------------------------------------------------------------------------------

\begin{block}{Materials}

\begin{columns} % Subdivide the first main column
\begin{column}{.54\textwidth} % The first subdivided column within the first main column
\begin{itemize}
\item Vestibulum nisl, quis euismod velit eros in ligula.
\begin{itemize}
\item Cras rhoncus quam et augue convallis in elementum urna tincidunt.
\end{itemize}
\item Proin ut vestibulum augue.
\begin{itemize}
\item Donec dapibus sagittis neque eu ultrices.
\end{itemize}
\end{itemize}
\end{column}

\begin{column}{.43\textwidth} % The second subdivided column within the first main column
\centering
\begin{figure}
\includegraphics[width=0.8\linewidth]{placeholder.jpg}
\caption{Figure caption}
\end{figure}
\end{column}
\end{columns} % End of the subdivision

\begin{itemize}
\item Curabitur sapien ligula, faucibus in feugiat quis, vestibulum a turpis.
\begin{itemize}
\item Phasellus quis nunc neque. Suspendisse mauris diam, suscipit non gravida in, placerat id enim. Ut nec ipsum in lectus ultrices sagittis.
\item Ut nec ipsum in lectus ultrices sagittis.
\item Phasellus quis nunc neque.
\end{itemize}
\end{itemize}

\end{block}

%----------------------------------------------------------------------------------------
%	METHODS
%----------------------------------------------------------------------------------------

\begin{block}{Methods}

\begin{itemize}
\item Maecenas Vel Nisl Elit
\begin{itemize}
\item Suspendisse potenti. Fusce a est eget turpis rhoncus varius sed sed dui. Cras justo nibh, bibendum a cursus eget, consequat et dui. Maecenas vel nisl elit, sed dignissim dolor. 
\item In hac habitasse platea dictumst.
\end{itemize}

\item Viewpoint Matching Constraints
\begin{itemize}
\item Cum sociis natoque penatibus et magnis dis parturient montes, nascetur ridiculus mus. 
\item Proin in nisi diam.
\item Nam ultricies pellentesque nunc, ultrices volutpat nisl ultrices a.
\end{itemize}

\item Volutpat 
\begin{itemize}
\item Duis semper lorem eget dui dignissim porttitor.
\item Nulla facilisi. In ullamcorper lorem quis dolor.
\end{itemize}
\end{itemize}

\end{block}

%----------------------------------------------------------------------------------------
%	MATHEMATICAL SECTION
%----------------------------------------------------------------------------------------

\begin{block}{Mathematical Section}

\begin{itemize}
\item Maecenas Ultricies Feugiat Velit Non Mattis.
\begin{itemize}
\item Duis ante erat, bibendum nec tempus nec, interdum quis est. Nulla at mollis tortor. Phasellus quis leo dolor, aliquam laoreet orci $X$ Donec dapibus sagittis neque eu nec, interdum quis est. $Y_n, n=1,\cdots,N$ ndum nec tempus nec, interd
\begin{align*}
X \rightarrow r(X) & = \arg \max_{c} \Big\{ \max_n \big\{ \sum_{x_i \in X} \delta(x_i,Y_{n,c})\big\} \Big\} 
\end{align*}
\item Cras faucibus scelerisque cursus. Proin ut vestibulum augue. $\delta(x_i,Y_{n,c})$
\end{itemize}
\item Fusce tempus arcu id ligula varius dictum. Donec ut nisl dui, ac consectetur elit. In nec enim porta augue venenatis sollicitudin. Phasellus quis nunc neque. Suspendisse mauris diam, suscipit non gravida in, placerat id enim. Ut nec ipsum in lectus ultrices sagittis.
\end{itemize}

\end{block}

%----------------------------------------------------------------------------------------

\end{column} % End of the first column

\begin{column}{.03\textwidth}\end{column} % Empty spacer column
 
\begin{column}{.465\textwidth} % The second column

%----------------------------------------------------------------------------------------
%	RESULTS
%----------------------------------------------------------------------------------------

\begin{block}{Results: Table}

\begin{itemize}
\item Ased Aliquet Luctus Lectus
\end{itemize}

\begin{table}
\begin{tabular}{l l l}
\toprule
\textbf{Treatments} & \textbf{Response 1} & \textbf{Response 2}\\
\midrule
Treatment 1 & 0.0003262 & 0.562 \\
Treatment 2 & 0.0015681 & 0.910 \\
Treatment 3 & 0.0009271 & 0.296 \\
\bottomrule
\end{tabular}
\caption{Table caption}
\end{table}

\begin{itemize}
\item Sollicitudin Vel Orci
\item Maecenas Ultricies Feugiat Velit Non Mattis.
\end{itemize}

\begin{table}
\begin{tabular}{l l l}
\toprule
\textbf{Treatments} & \textbf{Response 1} & \textbf{Response 2}\\
\midrule
Treatment 1 & 0.0003262 & 0.562 \\
Treatment 2 & 0.0015681 & 0.910 \\
Treatment 3 & 0.0009271 & 0.296 \\
\bottomrule
\end{tabular}
\caption{Table caption}
\end{table}
     
\end{block}

%------------------------------------------------

\begin{block}{Results: Figure}

\begin{figure}
\includegraphics[width=0.8\linewidth]{placeholder.jpg}
\caption{Figure caption}
\end{figure}

\end{block}

%----------------------------------------------------------------------------------------
%	CONCLUSION
%----------------------------------------------------------------------------------------

\begin{block}{Conclusion}

\begin{itemize}
\item Opet volutpat ligula. Duis semper lorem eget dui dignissim porttitor. Nulla facilisi. In ullamcorper lorem quis dolor iaculis nec egestas enim ultricies. Cras ut mauris elit, ut lacinia dui. Proin in ante et libero hendrerit iaculis.
\item Nulla eu erat a urna laoreet auctor id a turpis. Nam mollis tristique neque eu luctus. Suspendisse rutrum congue nisi sed convallis. 
\item Aenean id neque dolor.
\item Opet volutpat ligula. Duis semper lorem eget dui dignissim porttitor. Nulla facilisi. In ullamcorper lorem quis dolor iaculis nec egestas enim ultricies. Cras ut mauris elit, ut lacinia dui. Proin in ante et libero hendrerit iaculis.
\end{itemize}

\end{block}

%----------------------------------------------------------------------------------------
%	REFERENCES
%----------------------------------------------------------------------------------------

\begin{block}{References}
        
\nocite{*} % Insert publications even if they are not cited in the poster
\small{\bibliographystyle{unsrt}
\bibliography{sample}}

\end{block}

%----------------------------------------------------------------------------------------
%	ACKNOWLEDGEMENTS
%----------------------------------------------------------------------------------------

\begin{block}{Acknowledgments}

\begin{itemize}
\item Nam mollis tristique neque eu luctus. Suspendisse rutrum congue nisi sed convallis. Aenean id neque dolor. Pellentesque habitant morbi tristique senectus et netus et malesuada fames ac turpis egestas.
\end{itemize}

\end{block}

%----------------------------------------------------------------------------------------
%	CONTACT INFORMATION
%----------------------------------------------------------------------------------------

\setbeamercolor{block title}{fg=black,bg=orange!70} % Change the block title color

\begin{block}{Contact Information}

\begin{itemize}
\item Web: \href{http://www.stat.auckland.ac.nz/~brewer/}{http://www.stat.auckland.ac.nz/\textasciitilde brewer/}
\item Email: \href{mailto:bj.brewer@auckland.ac.nz}{bj.brewer@auckland.ac.nz}
\item Twitter: \href{http://www.twitter.com/brendonbrewer/}{@brendonbrewer}
\end{itemize}

\end{block}

%----------------------------------------------------------------------------------------

\end{column} % End of the second column

\begin{column}{.015\textwidth}\end{column} % Empty spacer column

\end{columns} % End of all the columns in the poster

\end{frame} % End of the enclosing frame

\end{document}

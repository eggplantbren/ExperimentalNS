\documentclass[a4paper, 11pt]{article}
\usepackage{graphicx}
\usepackage{natbib}
\usepackage{dsfont}
\usepackage[left=3cm,top=3cm,right=3cm]{geometry}

\renewcommand{\topfraction}{0.85}
\renewcommand{\textfraction}{0.1}
\parindent=0cm

\title{Nested sampling with multiple scalars}
\author{Brendon J. Brewer}
\date{\vspace{3cm}This work is licensed under the Creative Commons Attribution-ShareAlike 3.0 Unported License. To view a copy of this license, visit {\tt http://creativecommons.org/licenses/by-sa/3.0/deed.en\_GB}.}


\begin{document}
\maketitle

\abstract{Nested Sampling is an effective and popular
Monte Carlo algorithm for Bayesian computation. It is based on exploring the
prior $\pi(\mathbf{x})$ and successively imposing constraints on the value of
the likelihood $\mathcal{L}(\mathbf{x})$ that compress the prior mass by a
``known'' factor. This enables the calculation of the marginal likelihood
or evidence
$\mathcal{Z} = \int \pi(\mathbf{x})\mathcal{L}(\mathbf{x}) \, d^N\mathbf{x}$ and
properties of the posterior, or any other distribution that is intermediate
between the prior and the posterior. However, in some inference and
statistical mechanics problems, there are two or more scalar functions of
$\mathbf{x}$ that are relevant. For example, we might be interested in the
properties of
$p(\mathbf{x} | \lambda_1, \lambda_2) \propto \pi(\mathbf{x})\exp\left[-\lambda_1 f_1(\mathbf{x}) - \lambda_2 f_2(\mathbf{x})\right]$
for many different values of $\lambda_1$ and $\lambda_2$, including the
normalisations or ``partition function'' $\mathcal{Z}(\lambda_1, \lambda_2)$.
This work describes
progress towards solving this class of problems while maintaining the benefits
of Nested Sampling, such as the ability to cope with first-order phase
transitions.
}


If the prior is $\pi(\mathbf{x})$, then a family of canonical
probability distributions is given by
\begin{eqnarray}
p(\mathbf{x} | \lambda_1, ..., \lambda_n) = 
\frac{\pi(\mathbf{x})\exp(-\sum_{i=1}^n \lambda_i f_i(\mathbf{x}))}
{\mathcal{Z}(\lambda_1, ..., \lambda_n)}
\end{eqnarray}
This family of distributions arises from a MaxEnt update from the prior
$\pi(\mathbf{x})$ if the expected values of the functions
$\{f_i(\mathbf{x})\}$ are specified. In statistical mechanics it is useful to
quantify the properties of
$p(\mathbf{x} | \lambda_1, ..., \lambda_n)$ as a function of the
$\lambda$s.

\section{Nested sampling}



\section{Example: Potts model with some embellishments}
TBD

\end{document}


\documentclass[a4paper, 11pt]{article}
\usepackage{graphicx}
\usepackage{natbib}
\usepackage{dsfont}
\usepackage[left=3cm,top=3cm,right=3cm]{geometry}

\renewcommand{\topfraction}{0.85}
\renewcommand{\textfraction}{0.1}
\parindent=0cm

\title{Nested sampling with multiple scalars}
\author{Brendon J. Brewer}
\date{\vspace{3cm}This work is licensed under the Creative Commons Attribution-ShareAlike 3.0 Unported License. To view a copy of this license, visit {\tt http://creativecommons.org/licenses/by-sa/3.0/deed.en\_GB}.}

\begin{document}
\maketitle

If the prior is $\pi(\mathbf{x})$, then a family of canonical
probability distributions is given by
\begin{eqnarray}
p(\mathbf{x} | \lambda_1, ..., \lambda_n) = 
\frac{\pi(\mathbf{x})\exp(-\sum_{i=1}^n \lambda_i f_i(\mathbf{x}))}
{\mathcal{Z}(\lambda_1, ..., \lambda_n)}
\end{eqnarray}
This family of distributions arises from a MaxEnt update from the prior
$\pi(\mathbf{x})$ if the expected values of the functions
$\{f_i(\mathbf{x})\}$ are specified. In statistical mechanics it is useful to
quantify the properties of
$p(\mathbf{x} | \lambda_1, ..., \lambda_n)$ as a function of the
$\lambda$s.

\section{Nested sampling}



\section{Example: Potts model with some embellishments}
TBD

\end{document}

